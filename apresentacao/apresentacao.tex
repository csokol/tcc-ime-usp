\documentclass[brazil]{beamer}
\usepackage[brazilian]{babel}
\usepackage[utf8]{inputenc}

\input{pygments.tex}

\usetheme{Bergen}

\def\insertauthorindicator{}% Default is "Who?"
\def\insertinstituteindicator{}% Default is "From?"
\def\insertdateindicator{}% Default is "When?"

%\usecolortheme{beaver}

\title%(optional, only for long titles)
{MetricMiner}
\subtitle{Uma ferramenta
web de apoio à mineração de
repositórios de software}
\institute {
Orientador: Marco Aurélio Gerosa\\ 
Co-orientador: Mauricio Finavaro Aniche
}
\author
{Francisco Sokol}
\begin{document}

\frame{\titlepage}

	\section{Mineração de repositórios}

	\begin{frame}
		\frametitle{Mineração de repositórios}
		\begin{itemize}
			\item Estudo empírico da evolução de software 
			\item Aplicação de técnicas de data mining aos dados do histórico de desenvolvimento de um software
		\end{itemize}
	\end{frame}

	\begin{frame}
		\frametitle{Mineração de repositórios}
		\begin{itemize}
			\item Quais são as classes mais modificadas no projeto?
			\item Qual será a classe com mais bugs na próxima versão?
			\item Quais desenvolvedores devem trabalhar juntos?
		\end{itemize}
	\end{frame}

	\begin{frame}
		\frametitle{Motivação}
		Ferramentas atuais de mineração:
		\begin{itemize}
			\item Executam localmente
			\item Configuração complexa
			\item Recursos locais
			\item Pouca escalabilidade
		\end{itemize}
	\end{frame}

	\begin{frame}
		\frametitle{MetricMiner}
		Baseado no rEvolution \footnote{\url{http://github.com/mauricioaniche/revolution}}\vspace{0.5cm}

		Requisitos:
		\begin{itemize}
			\item Aplicação web
			\item Armazenamento de informações do sistema de controle de versão
			\item Cálculo de métricas de código
			\item Interface para consulta
		\end{itemize}
	\end{frame}

	\begin{frame}
		\frametitle{MetricMiner}
		\framesubtitle{Tecnologias}

		\begin{itemize}
			\item Java
			\begin{itemize}
				\item VRaptor
				\item Quartz Scheduler
			\end{itemize}
			\item MySQL
			\begin{itemize}
				\item Hibernate
			\end{itemize}
			\item Infraestrutura de cloud da Locaweb
		\end{itemize}
	\end{frame}

	\begin{frame}
		\frametitle{MetricMiner}
		\framesubtitle{Fila de execução}

		Tarefas de mineração são executadas \emph{assíncronamente}

		\begin{itemize}
			\item Download do repositório (Git e SVN)
			\item Processamento do sistema de controle de versão
			\item Cálculo de métricas de código
			\item Métricas de projeto
			\item Consulta em SQL aos dados
		\end{itemize}
	\end{frame}

	\begin{frame}
		\frametitle{MetricMiner}

		Criação de novas tarefas de mineração:

		\begin{scriptsize}
			\input{pygments/runnabletask.tex}
		\end{scriptsize}

	\end{frame}


	\begin{frame}
		\frametitle{MetricMiner}

		Criação de novas métricas de código:

		\begin{scriptsize}
			\begin{Verbatim}[commandchars=\\\{\}]
\PY{k+kd}{public} \PY{k+kd}{interface} \PY{n+nc}{MetricFactory} \PY{o}{\PYZob{}}
    \PY{k+kd}{public} \PY{n}{Metric} \PY{n+nf}{build}\PY{o}{(}\PY{o}{)}\PY{o}{;}
\PY{o}{\PYZcb{}}

\PY{k+kd}{public} \PY{k+kd}{interface} \PY{n+nc}{Metric} \PY{o}{\PYZob{}}
    \PY{n}{Collection}\PY{o}{\PYZlt{}}\PY{n}{MetricResult}\PY{o}{\PYZgt{}} \PY{n}{results}\PY{o}{(}\PY{n}{SourceCode} \PY{n}{source}\PY{o}{)}\PY{o}{;}
    \PY{k+kt}{void} \PY{n+nf}{calculate}\PY{o}{(}\PY{n}{InputStream} \PY{n}{is}\PY{o}{)}\PY{o}{;}
    \PY{k+kt}{boolean} \PY{n+nf}{matches}\PY{o}{(}\PY{n}{String} \PY{n}{name}\PY{o}{)}\PY{o}{;}
    \PY{n}{Class}\PY{o}{\PYZlt{}}\PY{o}{?}\PY{o}{\PYZgt{}} \PY{n}{getFactoryClass}\PY{o}{(}\PY{o}{)}\PY{o}{;}
\PY{o}{\PYZcb{}}

\PY{n+nd}{@MetricComponent}\PY{o}{(}\PY{n}{name}\PY{o}{=}\PY{l+s}{"Cyclomatic Complexity"}\PY{o}{)}
\PY{k+kd}{public} \PY{k+kd}{class} \PY{n+nc}{CCMetricFactory} \PY{k+kd}{implements} \PY{n}{MetricFactory} \PY{o}{\PYZob{}}
    \PY{k+kd}{public} \PY{n}{Metric} \PY{n+nf}{build}\PY{o}{(}\PY{o}{)} \PY{o}{\PYZob{}}
        \PY{k}{return} \PY{k}{new} \PY{n+nf}{CCMetric}\PY{o}{(}\PY{o}{)}\PY{o}{;}
    \PY{o}{\PYZcb{}}
\PY{o}{\PYZcb{}}
\end{Verbatim}

		\end{scriptsize}

	\end{frame}

	\begin{frame}
		\frametitle{MetricMiner}
		\framesubtitle{Métricas implementadas}

		Implementação com a biblioteca javaparser\footnote{\url{http://code.google.com/p/javaparser/}}

		\begin{itemize}
			\item Complexidade Ciclomática
			\item LCOM
			\item Fan Out
			\item Número de linhas
			\item Número de métodos
		\end{itemize}
	\end{frame}

	\begin{frame}
		\frametitle{Avaliação da ferramenta}
		\framesubtitle{Mineração da Apache Software Foundation}

		Mineração de mais de 300 projetos da Apache\footnote{\url{http://git.apache.org/}}

		\begin{itemize}
			\item 90 horas de duração
			\item 800 mil commits
			\item 2 mil autores
			\item 1.5 milhões de artefatos
			\item 5 milhões de versões de código
		\end{itemize}
	\end{frame}

	\begin{frame}
		\frametitle{Avaliação da ferramenta}
		\framesubtitle{Reprodução de estudo da literatura}

		Expansão de estudo publicado na área \cite{SoetensQUATIC2010}

		\vspace{0.5cm}

		\emph{Qual o efeito de refatorações sobre a complexidade de um sistema?}

	\end{frame}


	\begin{frame}
		\frametitle{Avaliação da ferramenta}
		\framesubtitle{Estudo original}

		\begin{itemize}
			\item Complexidade Ciclomática
			\item 700 versões do projeto PMD
			\item SVNKit + Eclipse Metrics
		\end{itemize}

	\end{frame}

	\begin{frame}
		\frametitle{Avaliação da ferramenta}
		\framesubtitle{Estudo original}
		\includegraphics[width=1.00\textwidth]{img/cc-soetens.png}
	\end{frame}

	\begin{frame}
		\frametitle{Avaliação da ferramenta}
		\framesubtitle{Estudo original}
		\includegraphics[width=1.00\textwidth]{img/tabela-soetens.png}
	\end{frame}

	\begin{frame}
		\frametitle{Avaliação da ferramenta}
		\framesubtitle{Análise dos autores}
		\begin{itemize}
			\item Poucas refatorações com remoção de código duplicado
			\item Commits com refatoração + nova funcionalidade
			\item Pequenas mudanças como movimentação de métodos e variáveis
		\end{itemize}
	\end{frame}

	\begin{frame}
		\frametitle{Avaliação da ferramenta}
		\framesubtitle{Reprodução do estudo}
		Consulta pela interface web\footnote{\url{http://metricminer.org.br/query/1}}
		Programa auxiliar para processar csv
		\begin{itemize}
			\item 250 projetos java
			\item 500 mil commits processados
		\end{itemize}
	\end{frame}

	\begin{frame}
		\frametitle{Avaliação da ferramenta}
		\framesubtitle{Reprodução do estudo}
		\begin{figure}[ht]
		\centering
		\includegraphics[width=0.7\textwidth]{img/camel.png}
		\end{figure}
	\end{frame}


	\begin{frame}
		\frametitle{Avaliação da ferramenta}
		\framesubtitle{Reprodução do estudo}
		\begin{figure}[ht]
		\centering
		\includegraphics[width=0.7\textwidth]{img/ant.png}
		\end{figure}
	\end{frame}

	\begin{frame}
		\frametitle{Avaliação da ferramenta}
		\framesubtitle{Reprodução do estudo}
		\begin{figure}[ht]
		\centering
		\includegraphics[width=0.7\textwidth]{img/tomcat.png}
		\end{figure}
	\end{frame}

	\begin{frame}
		\frametitle{Avaliação da ferramenta}
		\framesubtitle{Reprodução do estudo}
		\includegraphics[width=1.00\textwidth]{img/tabela-metricminer.png}
	\end{frame}

	\begin{frame}
		\frametitle{Avaliação da ferramenta}
		\framesubtitle{Conclusão}
		\begin{itemize}
			\item Maior quantidade e variedade de projetos analisados
			\item Resultados semelhantes ao do estudo original
			\item Processo de mineração mais simples e eficiente
		\end{itemize}
	\end{frame}

	\begin{frame}
		\frametitle{Trabalhos futuros}
		\begin{itemize}
			\item Paralelizar
			\item Usabilidade da interface web
			\item Outras métricas e linguagens
			\item Sistemas de bug tracking, listas de email
			\item API
		\end{itemize}
	\end{frame}

	\begin{frame}
		\centering \begin{Huge}\textbf Obrigado!\\ \vspace{0.5cm}\end{Huge}
		\centering \begin{huge}\textbf Perguntas?\\\end{huge}
		\vspace{0.7cm}
		\centering chico.sokol@gmail.com\\
		\centering \begin{scriptsize}\url{http://metricminer.org.br}\end{scriptsize}\\
		\centering \begin{scriptsize}\url{http://github.com/metricminer/metricminer}\end{scriptsize}\\
	\end{frame}


% etc

\bibliographystyle{sbc}
\bibliography{apresentacao}
\end{document}