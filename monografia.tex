%%%%%%%%%%%%%%%%%%%%%%%%%%%%%%%%%%%%%%%%%%%%%%%%%%%%%%%%%%
%%
%%                  Caelum Tubaina
%%
%%%%%%%%%%%%%%%%%%%%%%%%%%%%%%%%%%%%%%%%%%%%%%%%%%%%%%%%%%

\documentclass[a4paper, 12pt, twoside]{book}

%% Escrevendo em portugues:
\usepackage[brazil]{babel}	
\usepackage[T1]{fontenc}
\usepackage{float}
\usepackage[scaled]{helvet}
\usepackage[utf8]{inputenc}
\usepackage{upquote} %fix minted quotes
\usepackage{parskip} %fix paragraph spacing inside tubainabox
\usepackage{setspace}
\usepackage[a4paper, top=46mm, left=29mm, right=29mm, bottom=32mm, includefoot]{geometry}

\begin{document}

\begin{titlepage}
\vspace*{\fill}
\vspace{-5cm}
\center{\begin{spacing}{1.0}\huge \textbf {MetricMiner: uma ferramenta web de apoio à mineração de repositórios de software}\end{spacing}}
\vspace{2cm}
\center{\Large Francisco Sokol}
\vspace{1cm}
\center{\large \emph{Orientador}: Marco Aurélio Gerosa}
\center{\large \emph{Co-orientador}: Maurício Finavaro Aniche}
\vfill
\end{titlepage}

% \title{MetricMiner: uma ferramenta web de apoio à mineração de repositórios de software}
% \author{Francisco Sokol}
% \maketitle

\pagestyle{plain}
\pagenumbering{roman}
\setcounter{page}{1} 
\tableofcontents

\newpage

\pagenumbering{arabic}
\setcounter{page}{1}

\chapter{Introdução}

    \section{Contextualização}
        Evolução de Software é uma área da Engenharia de Software que estuda as atividades de desenvolvimento
        em um sistema de software após a sua concepção inicial e implantação em produção. Esse termo foi 
        usado pela primeira vez em um artigo publicado por Manny Lehman \cite{DBLP:series/springer/Mens08}. 
        Neste trabalho, o autor enuncia as ``leis da evolução de software'', defendendo que programas que 
        representam alguma atividade do mundo real evoluem continuamente, caso contrário, se tornam
        menos úteis e perdem seu valor \cite{Lehman1980b}. Além disso, Lehman afirma que este processo 
        de mudança contínua faz com que a complexidade do software cresça inevitavelmente, tornando sua 
        estrutura cada vez mais podre e seu custo de manutenção maior.
        
        Considereando ainda diversos trabalhos indicando que o custo de manutenção de um software 
        ultrapassa 50\% do custo total de um projeto, observa-se que encontrar meios de se
        manter a qualidade interna de um software é muito importante. Por meio do desenvolvimento 
        ferramentas e métodos, a evolução de software busca entender e controlar esse processo
        para o tornar mais eficiente.
        
        Nesse contexto, a Mineração de Repositórios de Software estuda esse processo de evolução de forma empírica, 
        por meio da análise dos artefatos envolvidos no seu desenvolvimento como código fonte, dados do sistema de
        controle de versão e sistemas de rastreamento de bugs \cite{Kagdi:2007}. 

    \section{Motivação}
    
    \section{Estrutura da monografia}

    
\chapter{Conceitos estudados}

    \section{Evolução de software}
    
    \section{Métricas de código}
    
    \section{Mineração de repositórios de software}

    \section{Trabalhos relacionados}
    
\chapter{Arquitetura do MetricMiner}

    \section{Tecnologias envolvidas}
    
    \section{Decisões arquiteturais}
    
    \section{Estendendo e contribuindo com o projeto}
    
\chapter{Avaliação da ferramenta}

    \section{Resultados obtidos com a mineração de repositórios públicos}

\chapter{Conclusão e trabalhos futuros}

\chapter{Parte subjetiva}

\bibliographystyle{sbc}
\bibliography{monografia}

\end{document}


