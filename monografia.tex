%%%%%%%%%%%%%%%%%%%%%%%%%%%%%%%%%%%%%%%%%%%%%%%%%%%%%%%%%%
%%
%%                  Caelum Tubaina
%%
%%%%%%%%%%%%%%%%%%%%%%%%%%%%%%%%%%%%%%%%%%%%%%%%%%%%%%%%%%


\documentclass[a4paper, 12pt, twoside]{book}

%% Escrevendo em portugues:
\usepackage[brazil]{babel}	
\usepackage[T1]{fontenc}
\usepackage{float}
\usepackage[scaled]{helvet}
\usepackage[utf8]{inputenc}
\usepackage{upquote} %fix minted quotes
\usepackage{parskip} %fix paragraph spacing inside tubainabox
\usepackage{setspace}
\usepackage{monografia}
\usepackage[a4paper, top=46mm, left=29mm, right=29mm, bottom=32mm, includefoot]{geometry}

\usepackage{fancyhdr}


\begin{document}

\begin{titlepage}
\vspace*{\fill}
\vspace{-5cm}
\center{\begin{spacing}{1.0}\huge \textbf{\textsc{MetricMiner: uma ferramenta web de apoio à mineração de repositórios de software}}\end{spacing}}
\vspace{2cm}
\center\textsc{{\Large Francisco Sokol}}
\vspace{1cm}
\center\textsc{{\large Orientador: Marco Aurélio Gerosa}}
\center\textsc{{\large Co-orientador: Maurício Finavaro Aniche}}
\vfill
\end{titlepage}

% \title{MetricMiner: uma ferramenta web de apoio à mineração de repositórios de software}
% \author{Francisco Sokol}
% \maketitle

\pagestyle{plain}
\pagenumbering{roman}
\setcounter{page}{1} 
\tableofcontents

\newpage

\pagenumbering{arabic}
\setcounter{page}{1}

\chapter{Introdução}

    \section{Contextualização}
        Evolução de Software é uma área da Engenharia de Software que estuda as atividades de desenvolvimento
        em um sistema de software após a sua concepção inicial e implantação em produção. Esse termo foi 
        usado pela primeira vez em um artigo publicado por Manny Lehman \cite{DBLP:series/springer/Mens08}. 
        Neste trabalho, o autor enuncia as ``leis da evolução de software'', defendendo que programas que 
        representam alguma atividade do mundo real evoluem continuamente, caso contrário, se tornam
        menos úteis e perdem seu valor \cite{Lehman1980b}. Além disso, Lehman afirma que este processo 
        de mudança contínua faz com que a complexidade do software cresça inevitavelmente, tornando sua 
        estrutura cada vez mais pobre e seu custo de manutenção maior.
        
        Considerando ainda diversos trabalhos indicando que o custo de manutenção de um software 
        ultrapassa 50\% do custo total de um projeto, observa-se que encontrar meios de se
        manter a qualidade interna de um software é muito importante. Por meio do desenvolvimento 
        ferramentas e métodos, a evolução de software busca entender e controlar esse processo
        para o tornar mais eficiente.
        
        Nesse contexto, a Mineração de Repositórios de Software estuda esse processo de evolução de forma empírica, 
        por meio da análise dos artefatos envolvidos no seu desenvolvimento como código fonte, dados do sistema de
        controle de versão e sistemas de rastreamento de bugs \cite{Kagdi:2007}. 

    \section{Motivação}
        Para desenvolver um trabalho em mineração de repositórios, o pesquisador é obrigado a carregar 
        diversos projetos em sua estação de trabalho e realizar uma série de cálculos sobre o código 
        dos projetos e sobre os metadados de seu repositório. Esse processo requer a instalação de diversas
        ferramentas e bibliotecas localmente para reaproveitar as ferramentas desenvolvidas nessa área,
        tornando o processo trabalhoso e demorado.

        Além de ser um processo complexo, esse tipo de pesquisa consome muitos recursos computacionais.
        Baixar os repositórios a serem minerados consome um volume considerável de banda. Depois, os 
        dados devem ser processados e persistidos em um banco de dados, ocupando um grande volume de 
        disco. Só então o pesquisador pode calcular métricas sobre esses dados, além de extrair relações
        entre os metadados do histórico do sistema de controle de versão, gastando uma quantidade grande 
        de processamento de CPU. Só depois de passar por todas essas etapas, é possível extrair dados e 
        avaliar hipóteses por meio de análises estatísticas.

        Dessas dificuldades surgiu a motivação para o desenvolvimento do MetricMiner, uma aplicação web
        que realiza todas as etapas da mineração de um repositório de software. Essa ferramenta disponibiliza
        um grande volume de dados já processados prontos para serem extraídos e analisados pelo pesquisador, 
        poupando tempo e recursos computacionais.
    
    \section{Estrutura da monografia}
        O restante desta monografia está estruturada da seguinte forma: 
        \begin{itemize}
            \item Seção \ref{ch:conceitos}: nesta seção são abordados temas 
                envolvidos no desenvolvimento desse trabalho, em um breve levantamento bibliográfico.
            \item Seção \ref{ch:arquitetura}: apresenta a arquitetura e as tecnologias
            envolvidas no desenvolvimento do MetricMiner.
            \item Seção \ref{ch:avaliacao}: expõe os resultados obtidos com a ferramenta 
                na mineração de repositórios de projetos de código aberto.
            \item Seção \ref{ch:conclusao}: análise dos resultados e levantamento 
                de possíveis extensões futuras do projeto.
            \item Seção \ref{ch:subjetiva}: apresenta as impressões do aluno na realização 
                desse trabalho e a sua relação com o curso de Bacharelado em Ciência da Computação.
        \end{itemize}
    
\chapter{Mineração de Repositórios de Software} \label{ch:conceitos}

    \section{Evolução de software}
    
    \section{Métricas de código}
    
    \section{Mineração de repositórios de software}

    \section{Trabalhos relacionados}
    
\chapter{Arquitetura do MetricMiner} \label{ch:arquitetura}

    \section{Tecnologias envolvidas}
    
    \section{Decisões arquiteturais}
    
    \section{Estendendo e contribuindo com o projeto}
    
\chapter{Avaliação da ferramenta} \label{ch:avaliacao}

    \section{Resultados obtidos com a mineração de repositórios públicos}

\chapter{Conclusão e trabalhos futuros} \label{ch:conclusao}

\chapter{Parte subjetiva} \label{ch:subjetiva}

\bibliographystyle{sbc}
\bibliography{monografia}

\end{document}


