%%%%%%%%%%%%%%%%%%%%%%%%%%%%%%%%%%%%%%%%%%%%%%%%%%%%%%%%%%
%%
%%                  Caelum Tubaina
%%
%%%%%%%%%%%%%%%%%%%%%%%%%%%%%%%%%%%%%%%%%%%%%%%%%%%%%%%%%%

\documentclass[a4paper, 12pt, twoside]{book}

%% Escrevendo em portugues:
\usepackage[brazil]{babel}	
\usepackage[T1]{fontenc}
\usepackage{float}
\usepackage[scaled]{helvet}
\usepackage[utf8]{inputenc}
\usepackage{upquote} %fix minted quotes
\usepackage{parskip} %fix paragraph spacing inside tubainabox

\begin{document}

\pagestyle{plain}
\pagenumbering{roman}
\setcounter{page}{1} 
\tableofcontents

\newpage

\pagenumbering{arabic}
\setcounter{page}{1}

\chapter{Introdução}

    \section{Contextualização}

    \section{Motivação}
    
    \section{Estrutura da monografia}

    
\chapter{Conceitos estudados}

    \section{Evolução de software}
    
    \section{Métricas de código}
    
    \section{Mineração de repositórios de software}

    
\chapter{Arquitetura da ferramenta e tecnologias envolvidas}

    \section{Tecnologias envolvidas}
    
    \section{Decisões arquiteturais}
    
    \section{Estendendo e contribuindo com o projeto}
    
    \section{Trabalhos relacionados}
    
\chapter{Avaliação da ferramenta}

    \section{Resultados obtidos com a mineração de repositórios públicos}

\chapter{Conclusão e trabalhos futuros}

\chapter{Parte subjetiva}

\end{document}


